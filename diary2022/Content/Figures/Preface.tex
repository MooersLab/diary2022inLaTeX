%!TEX root = ../main.tex
%%%%%%%%%%%%%%%%%%%%%% pref.tex %%%%%%%%%%%%%%%%%%%%%%%%%%%%%%%%%%%%%
%
% sample preface
%
% Use this file as a template for your own input.
%
%%%%%%%%%%%%%%%%%%%%%%%% Springer-Verlag %%%%%%%%%%%%%%%%%%%%%%%%%%

\preface
\index{preface}
%% Please write your preface here
Here come the golden words


%% Please "sign" your preface
\vspace{1cm}
\begin{flushright}\noindent
place(s),\hfill {\it First name  Surname}\\
month year\hfill {\it First name  Surname}\\
\end{flushright}


\section*{Known issues with typesetting this document}
\index{750words}

\subsection{Author index generation}
The documentclass does not support author indices directly.
The author index is formatted with a command called listofauthors.
This command is defined in the Preamble. 
I found this command online.
I tried to add the multicols environment.
The rightmost column overlapped with the header line on the first page of the index.
Hide this line might solve the problem.

\subsection{Underscores and the xurl package}
This package cannot handle underscores in URLs.
I obtained a memory error.
The solution was to comment out the xrul package.
The solution was provided by Dr. LainTze Lim of Overleaf.

\subsection{Stop overusing labels}
Add labels only when you need them, else they will be unreferenced and trigger a lot of warnings.
In addition, do not use labels in templates that write tex files unless you write only unique labels.
Otherwise, you will get a warning about multiple identical labels.
It is the situation with the month.tex files.
All twelve had an intro label.


\section*{Generation of daily record tex files for a year.}
\index{750words!daily page generation}
\index{setting up words for year!making pages}
The month.tex files for 2021 are found in the \url{~/db/750words2015Current} subdirectory of the db subdirectory.
They are in twelve separate folders labeled by month.

The daily files were generated by running a Python script \emph{makeDailyPagesVarB.py} in each folder.
This was done with a batch script. 

\begin{code}{}
\index{setting up words for year!making pages}
\label{lst750wordsMakingWords}
\caption{Contents of genPages.sh.}
\begin{minted}[frame=lines,framerule=2pt,linenos=false,xleftmargin=\parindent,breaklines]{bash}
#!/bin/bash
cd January
../makeDaily750PagesVerB.py 2021 'January' 1 31
cd ../February
../makeDaily750PagesVerB.py 2021 'February' 1 28
cd ../March
../makeDaily750PagesVerB.py 2021 'March' 1 31
cd ../April
../makeDaily750PagesVerB.py 2021 'April' 1 30
cd ../May
../makeDaily750PagesVerB.py 2021 'May' 1 31
cd ../June
../makeDaily750PagesVerB.py 2021 'June' 1 30
cd ../July
../makeDaily750PagesVerB.py 2021 'July' 1 31
cd ../July
../makeDaily750PagesVerB.py 2021 'July' 1 31
cd ../August
../makeDaily750PagesVerB.py 2021 'August' 1 31
cd ../September
../makeDaily750PagesVerB.py 2021 'September' 1 30
cd ../October
../makeDaily750PagesVerB.py 2021 'October' 1 31
cd ../November
../makeDaily750PagesVerB.py 2021 'November' 1 30
cd ../December
../makeDaily750PagesVerB.py 2021 'December' 1 31
\end{minted}
\end{code}


RAM may limit 750words to one document per year.


\section*{Correcting monthly.tex and making daily.tex files}
\index{latex!updating tex files!sed}

I used colons as delimiters in the following sed commands. 

\begin{code}{}
\index{setting up words for year!correcting existing files}
\label{lst:750words}
\caption{Correcting existing files.}
\begin{minted}[frame=lines,framerule=2pt,linenos=false,xleftmargin=\parindent,breaklines]{bash}
find . -name "February2021.tex" -type f | xargs  sed -i -e 's:2020:2021:g'
find . -name "February2021.tex" -type f | xargs  sed -i -e 's:./February:./Content/February:g'
find . -name "./Content/October/October2021.tex" -type f | xargs  sed -i -e 's:./October:./Content/October:g'
find . -name "./Content/November/November2021.tex" -type f | xargs  sed -i -e 's:./November:./Content/November:g'
find . -name "./Content/December/December2021.tex" -type f | xargs  sed -i -e 's:./December:./Content/December:g'
\end{minted}
\end{code}